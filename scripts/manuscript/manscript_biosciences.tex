\documentclass[3p]{elsarticle} %review=doublespace preprint=single 5p=2 column
%%% Begin My package additions %%%%%%%%%%%%%%%%%%%
\usepackage[hyphens]{url}



\usepackage{lineno} % add

\usepackage{graphicx}
%%%%%%%%%%%%%%%% end my additions to header

\usepackage[T1]{fontenc}
\usepackage{lmodern}
\usepackage{amssymb,amsmath}
\usepackage{ifxetex,ifluatex}
\usepackage{fixltx2e} % provides \textsubscript
% use upquote if available, for straight quotes in verbatim environments
\IfFileExists{upquote.sty}{\usepackage{upquote}}{}
\ifnum 0\ifxetex 1\fi\ifluatex 1\fi=0 % if pdftex
  \usepackage[utf8]{inputenc}
\else % if luatex or xelatex
  \usepackage{fontspec}
  \ifxetex
    \usepackage{xltxtra,xunicode}
  \fi
  \defaultfontfeatures{Mapping=tex-text,Scale=MatchLowercase}
  \newcommand{\euro}{€}
\fi
% use microtype if available
\IfFileExists{microtype.sty}{\usepackage{microtype}}{}
\bibliographystyle{elsarticle-harv}
\ifxetex
  \usepackage[setpagesize=false, % page size defined by xetex
              unicode=false, % unicode breaks when used with xetex
              xetex]{hyperref}
\else
  \usepackage[unicode=true]{hyperref}
\fi
\hypersetup{breaklinks=true,
            bookmarks=true,
            pdfauthor={},
            pdftitle={The promise of U.S. private lands for reaching 21st century conservation targets},
            colorlinks=false,
            urlcolor=blue,
            linkcolor=magenta,
            pdfborder={0 0 0}}
\urlstyle{same}  % don't use monospace font for urls

\setcounter{secnumdepth}{0}
% Pandoc toggle for numbering sections (defaults to be off)
\setcounter{secnumdepth}{0}


% tightlist command for lists without linebreak
\providecommand{\tightlist}{%
  \setlength{\itemsep}{0pt}\setlength{\parskip}{0pt}}


% Pandoc citation processing
\newlength{\cslhangindent}
\setlength{\cslhangindent}{1.5em}
\newlength{\csllabelwidth}
\setlength{\csllabelwidth}{3em}
\newlength{\cslentryspacingunit} % times entry-spacing
\setlength{\cslentryspacingunit}{\parskip}
% for Pandoc 2.8 to 2.10.1
\newenvironment{cslreferences}%
  {}%
  {\par}
% For Pandoc 2.11+
\newenvironment{CSLReferences}[2] % #1 hanging-ident, #2 entry spacing
 {% don't indent paragraphs
  \setlength{\parindent}{0pt}
  % turn on hanging indent if param 1 is 1
  \ifodd #1
  \let\oldpar\par
  \def\par{\hangindent=\cslhangindent\oldpar}
  \fi
  % set entry spacing
  \setlength{\parskip}{#2\cslentryspacingunit}
 }%
 {}
\usepackage{calc}
\newcommand{\CSLBlock}[1]{#1\hfill\break}
\newcommand{\CSLLeftMargin}[1]{\parbox[t]{\csllabelwidth}{#1}}
\newcommand{\CSLRightInline}[1]{\parbox[t]{\linewidth - \csllabelwidth}{#1}\break}
\newcommand{\CSLIndent}[1]{\hspace{\cslhangindent}#1}

\journal{TBD}
\graphicspath{{ext-figs/}}
\usepackage{float}
\linenumbers
\usepackage{setspace}\doublespacing



\begin{document}


\begin{frontmatter}

  \title{The promise of U.S. private lands for reaching 21st century
conservation targets}
    \author[espm]{Melissa Chapman\corref{Corresponding Author}}
  
    \author[espm]{Carl Boettiger}
  
    \author[espm]{Justin Brashares}
  
      \address[espm]{Dept. of Environmental Science, Policy, and
Management, University of California Berkeley, Berkeley, CA, USA}
    
  \begin{abstract}
  Coincident with an international movement to protect 30\% of global
  land and sea over the next decade, the United States has committed to
  more than doubling its current protected land area by 2030. While
  publicly owned and managed protected areas have been the cornerstone
  of area-based conservation in the United States and globally over the
  past century, such areas are costly to establish and have shown
  limited capacity to protect areas of highest value for biodiversity
  protection and climate mitigation. Here we examine the potential
  contribution of private lands to reach biodiversity and climate
  mitigation targets in the United States. Using the largest national
  database of public and private land conservation alongside
  distributions of biodiversity priority areas, current and projected
  species richness and carbon sequestration for North America, we show
  that private land conservation in the form of conservation easements
  has been more effective than federal protected areas in targeting
  areas of high value for biodiversity and climate change mitigation. We
  also explore how private lands are more commonly in areas designated
  as high conservation priority, hold significantly higher species
  richness than public lands and also hold more vulnerable carbon per
  unit area. We expand on these results to explore the critical role,
  and potential pitfalls, of leveraging private land conservation to
  meet post-2020 conservation goals and reducing biodiversity loss in
  the United States.
  \end{abstract}
   \begin{keyword} 30x30, biodiversity, conservation
easements, conservation targets, protected areas, climate
mitigation\end{keyword}
 \end{frontmatter}

\hypertarget{introduction}{%
\section{Introduction}\label{introduction}}

Following another decade of accelerating biodiversity
loss,\textsuperscript{1} the Convention on Biological Diversity (CBD) is
promoting a post-2020 transnational agreement on biodiversity
conservation. Largely coalesced around the promise of protecting 30\% of
the Earth's land and sea by 2030 (``30x30''), this agreement will
influence the next decade of global conservation policies and
biodiversity outcomes.\textsuperscript{2--4} In hopes of not repeating
the shortcomings of past area-based conservation
agreements,\textsuperscript{1} scientists and policymakers have
emphasized modern definitions of area-based protection that recognize
the importance of private and working land contributions to meeting
biodiversity and climate mitigation goals.\textsuperscript{2,5}

The United States is among the first countries to pass a legal mandate
in response to early drafts of the post-2020 CBD biodiversity
targets.\textsuperscript{4} In a 2021 Executive Order on ``Tackling the
Climate Crisis at Home and Abroad'', the Biden administration committed
to conserving 30\% of United States lands and waters by the year 2030,
with the broader goals of safeguarding food production and biodiversity
while mitigating climate change (Exec. Order No.~14008, 2021). With less
than 15\% of current US lands permanently protected in areas managed for
biodiversity,\textsuperscript{6} meeting this target will require an
unprecedented expansion of land protection over the next decade. While
protected areas owned by federal agencies account for the majority of
protected land in the U.S., they are legally cumbersome to implement
aside from those established under the Antiquities Act (i.e., National
Monuments). Moreover, despite the increasing prevalence of spatial
conservation planning and conservation
prioritization,\textsuperscript{7,8} several studies suggest protected
areas created to date overlap poorly with priority areas for
biodiversity conservation.\textsuperscript{2,9}

In an attempt to meet ambitious area-based targets while simultaneously
reducing potential mismatches between lands managed for biodiversity and
biodiversity distributions themselves, both federal and sub-national
post-2020 legislation and proposed pathways to meeting the legislative
targets in the U.S. have emphasized broader engagement with private and
working land. However, studies exploring the mismatch of protected areas
and biodiversity have largely ignored how other area-based conservation
measures, such as private land conservation, align with areas of high
conservation priority.\textsuperscript{2,9} Without a systematic
understanding of the relative capacity of private land conservation to
target key biodiversity areas and opportunities for climate change
mitigation, it is difficult to assess if the emphasis on private lands
is a well-informed policy direction for expanding area-based
conservation.

Private land protection measures, including private reserves, land
trusts, and conservation easements,have long contributed to land
conservation in the United States despite representing only a small
fraction of the total land under protection.\textsuperscript{10} While
private land conservation takes many forms, conservation easements -
voluntary legal agreements that permanently limit the uses of private
land to protect conservation values - have garnered particular interest
from conservation initiatives in the U.S. and elsewhere, due to their
cost-efficacy and legal flexibility.\textsuperscript{11} While a large
body of literature has examined drivers and impacts of conservation
easement adoption,\textsuperscript{12} management
attributes,\textsuperscript{13} and efficacy,\textsuperscript{14}
quantifying the value of conservation easements for biodiversity at a
national scale has been impeded by a lack of centralized data on parcel
delineations. Fortunately, new products such as the U.S. National
Conservation Easements Database (NCED;)\textsuperscript{15} now provide
opportunities to visualize and analyze the relative efficacy of private
land conservation measures in targeting areas of high conservation
value.

Here, we used the national compilation of spatial data on conservation
easements (NCED) to quantify the value of existing U.S. easements for
protecting biodiversity and securing carbon. Using the
NCED\textsuperscript{15} alongside distributions of biodiversity
priority areas (8), current species richness (15; 16), projected species
richness under climate change (17), and carbon sequestration in North
America (18), we assessed the conservation value of easements in the
United States relative to federal protected areas and unprotected lands.
We also tested if protected areas and conservation easements created in
the last 20 years (2001-2020) showed increased targeting of priority
areas for biodiversity conservation or climate mitigation. Taken
together, our analyses explore the potential of private lands to
complement traditional protected area contributions to meeting
qualitative elements of 2030 conservation targets, such as climate
change mitigation and climate resilience.

\hypertarget{conservation-in-key-biodiversity-areas}{%
\subsection{Conservation in key biodiversity
areas}\label{conservation-in-key-biodiversity-areas}}

Conservation easements managed for biodiversity (GAP 1 and GAP 2)
account for a significantly smaller total area than equivalently managed
federal protected areas (Fig. \ref{fig:fig1}B). Additionally,
conservation easements are on average significantly smaller per
management unit than protected areas (Fig. \ref{fig:fig1}C). Over the
past 20 years, conservation easements have increased in their rate of
adoption relative to protected areas (Fig. \ref{fig:fig1}D). While
conservation easements are typically smaller than protected areas, they
are more likely to overlap with land identified as a biodiversity
priority (Fig. \ref{fig:fig1}E).

Both GAP 1 and 2 protected areas and conservation easements have higher
mean species richness values than background U.S. lands (all lands
within U.S. borders), but lower mean richness values than all private
lands (estimated as all lands not included in PAD-US Fee GAP 1-4;
Methods and Materials) (Fig. \ref{fig:fig2}). Notably, GAP 1 and 2
easements have higher mean species richness than GAP 1 and 2 protected
areas. This holds true across aggregate richness as well as birds, fish,
and mammal richness, but is not true of amphibians or reptile richness
alone (Fig. \ref{fig:fig2}). Overall, public lands (GAP 1-4) have
significantly lower richness values across all taxa compared to private
lands and compared to total background values across all U.S. lands.
This holds true for all taxa (Fig. \ref{fig:fig2}). However, when
looking only at vulnerable, endangered, and critically endangered
(CRENVU) species, as well as small range species, protected areas have
higher mean richness values compared to conservation easements
(Supplemental information; Fig. S2). The patterns of private and public
land distributions relative to species richness distributions have
remained relatively constant across the past two decades (Fig.
\ref{fig:fig3} and Fig. S1).

\hypertarget{climate-resilient-biodiversity-conservation-and-land-based-climate-change-mitigation}{%
\subsection{Climate-resilient biodiversity conservation and land-based
climate change
mitigation}\label{climate-resilient-biodiversity-conservation-and-land-based-climate-change-mitigation}}

Under future climate change scenarios (high emissions: RCP 8.5
\cite{Lawler2020b}), conservation easements and protected areas closely
track projected background mean species richness values across all U.S.
land (Fig. \ref{fig:fig4}A). Notably, protected areas and conservation
easements (GAP 1 and GAP 2) had very similar mean future richness
values. While conservation easements have marginally improved their
tracking of future richness patterns over the past decade (Fig.
\ref{fig:fig4}C), protected areas have not (Fig. \ref{fig:fig4}C).

Contributions to nature-based climate mitigation also varied
significantly across protected areas and conservation easements.
Unsurprisingly, given their larger land area, protected areas accounted
for significantly more total above and below ground carbon (Fig.
\ref{fig:fig5}A). However, conservation easements had higher above
ground carbon on a per unit area basis (Fig. \ref{fig:fig5}B).

\hypertarget{discussion}{%
\section{Discussion}\label{discussion}}

Doubling the area of protected land in the United States over the next
decade while also prioritizing land with high biodiversity and climate
mitigation value will require significant investment in, and expansion
of, private land conservation measures. We show that private land
conservation instruments (conservation easements) better target areas
with high conservation value (Fig. 1E), high species richness (Fig. 2)
and high climate mitigation potential (Fig. 5) relative to
federally-owned protected areas managed for biodiversity across the U.S.
Importantly, our calculation of the average conservation value of public
and private lands shows that private lands hold the majority of
currently unprotected land with high biodiversity and climate mitigation
value (Fig. 2 and Fig. 5). The urgency of expanding land protection to
halt biodiversity loss will require flexible and expedient pathways to
implementing protections on these lands. Meeting 30\% area targets by
2030 will demand conservation actions that complement the historically
unjust and legally cumbersome processes of implementing new national
parks. Conservation easements and other forms of private land protection
provide compelling and cost-effective alternatives.

\#\#Protecting key biodiversity areas Area-based conservation goals risk
incentivizing the protection of cost-effective and opportunistically
available land rather than land with high conservation and climate
mitigation value (21). We find that unprotected private land is
distributed in areas with higher mean species richness values than
public land that is not managed for biodiversity. Similarly,
conservation easements more effectively target areas with high species
richness than public protected areas (Fig. 2). However, we find that
neither public protected areas nor conservation easements have
significantly improved their targeting of species richness over the past
two decades (Fig. 3) despite the expansion of spatial biodiversity data
(7) and the widely accepted Aichi Biodiversity Targets of the previous
decade.

While species richness is only one component of biodiversity, it is a
commonly used proxy to prioritize and assess the distribution of
protection relative to key biodiversity areas (22). Exploring
biodiversity metrics such as functional and phylogenetic diversity, as
well as other considerations commonly used in planning reserve networks
such as complementarity and endemism, will be critical to prioritizing
future investment in both private and public protected areas. Notably,
more than half of threatened and endangered species rely on private land
for critical habitat (U.S. Fish and Wildlife Service, 1997). However,
despite this reliance of threatened and endangered species on private
lands, we found that the distributions of endangered, vulnerable, and
small range species more closely track protected areas than conservation
easements (SI Fig. S2), highlighting the importance of complementary
approaches to land protection.

\hypertarget{climate-resilience-and-mitigation-potential-on-private-lands}{%
\subsection{Climate resilience and mitigation potential on private
lands}\label{climate-resilience-and-mitigation-potential-on-private-lands}}

As conservation practitioners decide where and how to protect land,
considering the potential impacts of climate-driven species range shifts
is critical to ensure resilient networks of protected lands over the
next decade. Examples of misguided land conservation due to shifting
ranges of critical species are plentiful (23). Our analysis shows that
both protected areas and conservation easements were less targeted
towards lands with high species richness under climate change (Fig. 4)
compared to richness in current climate conditions (Fig. 2), suggesting
that climate resilient biodiversity conservation will require more
effective prioritization of lands that are projected to be important for
biodiversity. Similar to our analysis of current species richness
distributions, private land held the highest density of projected future
species richness overall, and thus should be central in to designing
climate resilient pathways to achieving 30\% national protection. While
designing climate resilient biodiversity protections is important given
current emissions trajectories, climate mitigation is critical to
slowing climate change (24) and its impact on biodiversity (25; 26).
Land-based climate mitigation pathways (among other emissions reductions
pathways) are a central component of post-2020 area-based conservation
targets (Exec. Order No.~14008, 2021). Unsurprisingly, conservation
easements accounted for a significantly smaller portion of total above
and below ground carbon than protected lands due to being only a
fraction of the area of fee-owned protected areas (Fig. 5A). However, we
found that conservation easements store significantly more above ground
carbon than protected areas on a per unit area basis (Fig. 5B). We also
found that private lands overall held the majority of land carbon in the
U.S. (Supporting information; Fig. S3). Thus, these lands hold the
greatest potential for significant progress towards land-based climate
mitigation.

\hypertarget{avoiding-pitfalls-of-private-land-conservation}{%
\subsection{Avoiding pitfalls of private land
conservation}\label{avoiding-pitfalls-of-private-land-conservation}}

Despite the promise of private land contributions to biodiversity
protection and climate mitigation, conservation easements and other
private land protection measures have been criticized for ineffective
management and monitoring, as well as inequitable access and outcomes.
Private land protections are often opaque in their implemented
management practices, particularly when compared to publicly managed
lands (27). Furthermore, monitoring the impact of management practices
on private land at a national scale is difficult and disjointed.
Systematic monitoring of private lands will necessarily raise concerns
of privacy, potentially dissuading adoption of agreements in key areas.
Further, private land conservation measures, including conservation
easements, may disproportionately benefit high income landowners, often
limit public access, and are rooted in legacies of racial capitalism and
environmental injustice (28). Mitigating these issues through broader
community engagement, locally-defined monitoring protocols, and
increasing public access will be critical to ensuring private land
conservation contributes to the equity and access targets of post-2020
conservation goals. Finally, it is notable that conservation easements
typically conserve smaller parcels than protected areas (Fig. 1C),
potentially resulting in patchier landscapes and increasing the impact
of edge effects (29). However, categorizing parcels of protection as
either ``small and targeted'' or ``large and mismatched'' is a false
dichotomy -- parcel size of either conservation easements or protected
areas is not correlated with species richness in the U.S. (Supporting
information; Fig. S4). Even when accounting for area and state of
protected areas and easements, easements had significantly higher
richness values on a per parcel basis. Still, smaller parcels are likely
to be more common in private land protections due to land ownership
patterns in the United States. Thus, strategies to spatially cluster
easements in high priority areas may help ameliorate edge effects and
improve connectivity.

\hypertarget{sub-national-governance-and-private-land-conservation}{%
\subsection{Sub-national governance and private land
conservation}\label{sub-national-governance-and-private-land-conservation}}

While our analysis focused on private land conservation distributions at
a national scale, development and implementation of 30x30 legislation in
the United States (and likely in other federalist countries) will
largely be driven by sub-national governing bodies (4). On the
sub-national scale in the U.S., private land protections have already
been featured in a number of state-based 30x30 executive orders. A
deeper exploration of the sub-national distribution of private and
public land relative to biodiversity and carbon distributions will be
critical to ensuring that policies align with the resources in a given
governance unit, rather than assuming national scale patterns are
relevant at smaller scales (30). While accounting for State in our
analysis does not change the qualitative finding that easements better
target areas of higher species richness (Table S3), comparative analyses
will also be critical to understanding sociopolitical and ecological
contexts that impact the value of easement to meeting large-scale
conservation targets. Investigating differences in the conservation
value of public and private lands across sub-national scales of
governance may also help clarify the mechanisms driving the patterns of
private and public land protections on the national scale. Additionally,
understanding the structure of private land initiatives or
public-private partnerships that are actively working towards spatial
coordination of protection and biodiversity will be central to improving
the targeting of protection over the next decade.

\hypertarget{conclusion}{%
\section{Conclusion}\label{conclusion}}

Our analysis provides a national scale comparison of public and private
lands conservation in the United States and highlights the importance of
private land conservation for climate resilient biodiversity protection.
We show that private conservation is among the most effective and
feasible land-based pathways to meeting U.S. land-based climate change
mitigation goals by 2030. Despite numerous transnational and national
environmental initiatives over the past fifty years, biodiversity loss,
land conversion and climate change continue at unprecedented rates (31;
32). Meeting post-2020 biodiversity targets will require policy that
synergistically expands biodiversity protection on both private and
public lands while targeting areas of high conservation and climate
mitigation value.

\hypertarget{methods}{%
\section{Methods}\label{methods}}

\hypertarget{data}{%
\subsubsection{Data}\label{data}}

We acquired protected area and conservation easement delineations from
the United States Protected Area Database (PAD-US) (19). PAD-US compiles
conservation easement data from the National Conservation Easements
(NCED) (14) which contains over 130,000 easements (an estimated 60\% of
all U.S. easements; sensitivity analysis of results to missing data
available in Supporting Information). We restricted our analysis of
``protected areas'' to land administered by public agencies (fee-owned)
and managed for biodiversity (GAP 1 and GAP 2; Table S1). Similarly, we
include only conservation easements that are managed for biodiversity
(classified as GAP 1 or GAP 2) in the analysis of ``protected'' private
land. Hereafter, we refer to these two categories of land designations
as simply ``protected areas'' and ``conservation easements''. Protected
areas and conservation easements with invalid or missing geometries in
the PAD-US dataset were excluded from the study. Our final dataset
included 2579 protected areas and 1297 conservation easements managed
under GAP 1 criteria (fully protected and allowing only for natural
disturbances), and 313269 protected areas and 29351 conservation
easements under GAP 2 criteria (fully protected and allowing for
management action) (Fig. 1A; Table S1). We compared biodiversity and
climate mitigation values in our set of GAP 1 and 2 protected areas and
conservation easements with those of all federally owned public lands
and all lands held in private ownership. For those analyses, we defined
public lands as any land in the ``fee-owned'' PAD-US database
(regardless of GAP status). All other lands were considered ``private''.

Biodiversity priority areas were delineated using land in the 10th
percentile of biodiversity priority index values in the United States
(details on biodiversity priority indices can be found in (8)). Current
species richness, CRENVU richness, and Range-size rarity was estimated
using IUCN data, and calculated using raw IUCN ranges (version 2017-3)
for amphibians, birds and mammals (CITE). While there are a number of
alternative methods for mapping species richness (e.g., 17), there is no
evidence to suggest that range maps would be systematically biased
towards one given land protection measure over another. We calculated
future species richness using projected range distributions from Lawler
et al.~(2020) (17). Future ranges were estimated for each species under
three separate high emissions (RCP 8.5) climate change scenarios (17).
We approximated future richness as the number of species in a given
pixel (5 km2 resolution) using the mean of all three climate scenarios.
To assess climate change mitigation values of lands across management
types, we used vulnerable carbon maps, which estimate the carbon that
would be lost under a land conversion event. (18).

\hypertarget{analysis}{%
\subsubsection{Analysis}\label{analysis}}

We calculated mean species richness values for current and future
species distributions across public and private management units in R
(Supporting Information). Main figures represent overall differences in
biodiversity metrics and carbon density (area-weighted means across all
protected parcels). Differences in mean richness values across
individual protected areas and conservation easements were assessed
using t-tests (Supporting Information). We used propensity score
matching to estimate the average marginal difference of mean species
richness between conservation easements and protected areas accounting
for the confounding effect of area of parcels and subnational governance
(state) (Supporting Information). Mean vulnerable carbon values per
polygon were calculated using the same methods as above.

\hypertarget{refs}{}
\begin{CSLReferences}{0}{0}
\leavevmode\vadjust pre{\hypertarget{ref-Buchanan2020a}{}}%
\CSLLeftMargin{1. }
\CSLRightInline{Buchanan, G. M., Butchart, S. H. M., Chandler, G. \&
Gregory, R. D. {Assessment of national-level progress towards elements
of the Aichi Biodiversity Targets}. \emph{Ecological Indicators} (2020)
doi:\href{https://doi.org/10.1016/j.ecolind.2020.106497}{10.1016/j.ecolind.2020.106497}.}

\leavevmode\vadjust pre{\hypertarget{ref-Maxwell2020a}{}}%
\CSLLeftMargin{2. }
\CSLRightInline{Maxwell, S. L. \emph{et al.} {Area-based conservation in
the twenty-first century}. (2020)
doi:\href{https://doi.org/10.1038/s41586-020-2773-z}{10.1038/s41586-020-2773-z}.}

\leavevmode\vadjust pre{\hypertarget{ref-Tsioumani2020}{}}%
\CSLLeftMargin{3. }
\CSLRightInline{Tsioumani, E. {Convention on Biological Diversity: A
Review of the Post-2020 Global Biodiversity Framework Working Group
Negotiations}. (2020)
doi:\href{https://doi.org/10.3233/EPL-200207}{10.3233/EPL-200207}.}

\leavevmode\vadjust pre{\hypertarget{ref-Diversity2020}{}}%
\CSLLeftMargin{4. }
\CSLRightInline{Biological Diversity, C. on. {UPDATE OF THE ZERO DRAFT
OF THE POST-2020 GLOBAL BIODIVERSITY FRAMEWORK}. (2020).}

\leavevmode\vadjust pre{\hypertarget{ref-Marvier2014}{}}%
\CSLLeftMargin{5. }
\CSLRightInline{Marvier, M. \& Kareiva, P. {The evidence and values
underlying 'new conservation'}. (2014)
doi:\href{https://doi.org/10.1016/j.tree.2014.01.005}{10.1016/j.tree.2014.01.005}.}

\leavevmode\vadjust pre{\hypertarget{ref-GAP2018}{}}%
\CSLLeftMargin{6. }
\CSLRightInline{(GAP), U. S. G. S. (USGS). G. A. P. \emph{{Protected
Areas Database of the United States (PAD-US) 2.1: U.S. Geological Survey
data release}}. (2018) doi:\url{https://doi.org/10.5066/P955KPLE}.}

\leavevmode\vadjust pre{\hypertarget{ref-McIntosh2017}{}}%
\CSLLeftMargin{7. }
\CSLRightInline{McIntosh, E. J., Pressey, R. L., Lloyd, S., Smith, R. J.
\& Grenyer, R. {The Impact of Systematic Conservation Planning}. (2017)
doi:\href{https://doi.org/10.1146/annurev-environ-102016-060902}{10.1146/annurev-environ-102016-060902}.}

\leavevmode\vadjust pre{\hypertarget{ref-Sinclair2018a}{}}%
\CSLLeftMargin{8. }
\CSLRightInline{Sinclair, S. P. \emph{et al.} {The use, and usefulness,
of spatial conservation prioritizations}. \emph{Conservation Letters}
(2018)
doi:\href{https://doi.org/10.1111/conl.12459}{10.1111/conl.12459}.}

\leavevmode\vadjust pre{\hypertarget{ref-Jenkins2015}{}}%
\CSLLeftMargin{9. }
\CSLRightInline{Jenkins, C. N., Van Houtan, K. S., Pimm, S. L. \&
Sexton, J. O. {US protected lands mismatch biodiversity priorities}.
\emph{Proceedings of the National Academy of Sciences of the United
States of America} (2015)
doi:\href{https://doi.org/10.1073/pnas.1418034112}{10.1073/pnas.1418034112}.}

\leavevmode\vadjust pre{\hypertarget{ref-WALLACE2008a}{}}%
\CSLLeftMargin{10. }
\CSLRightInline{Wallace, G. N. {Land trusts, private reserves and
conservation easements in the United States}. \emph{Conservation biology
: the journal of the Society for Conservation Biology} (2008)
doi:\href{https://doi.org/10.3375/0885-8608(2008)28\%5B109:CMAMAO\%5D2.0.CO;2}{10.3375/0885-8608(2008)28{[}109:CMAMAO{]}2.0.CO;2}.}

\leavevmode\vadjust pre{\hypertarget{ref-CortesCapano2019}{}}%
\CSLLeftMargin{11. }
\CSLRightInline{Cortés Capano, G., Toivonen, T., Soutullo, A. \& Di
Minin, E. {The emergence of private land conservation in scientific
literature: A review}. (2019)
doi:\href{https://doi.org/10.1016/j.biocon.2019.07.010}{10.1016/j.biocon.2019.07.010}.}

\leavevmode\vadjust pre{\hypertarget{ref-Stroman2017}{}}%
\CSLLeftMargin{12. }
\CSLRightInline{Stroman, D. A., Kreuter, U. P. \& Gan, J. {Balancing
Property Rights and Social Responsibilities: Perspectives of
Conservation Easement Landowners}. \emph{Rangeland Ecology and
Management} (2017)
doi:\href{https://doi.org/10.1016/j.rama.2016.11.001}{10.1016/j.rama.2016.11.001}.}

\leavevmode\vadjust pre{\hypertarget{ref-Rissman2007}{}}%
\CSLLeftMargin{13. }
\CSLRightInline{Rissman, A. R. \emph{et al.} {Conservation easements:
Biodiversity protection and private use}. \emph{Conservation Biology}
(2007)
doi:\href{https://doi.org/10.1111/j.1523-1739.2007.00660.x}{10.1111/j.1523-1739.2007.00660.x}.}

\leavevmode\vadjust pre{\hypertarget{ref-Merenlender2004}{}}%
\CSLLeftMargin{14. }
\CSLRightInline{Merenlender, A. M., Huntsinger, L., Guthey, G. \&
Fairfax, S. K. {Land Trusts and Conservation Easements: Who Is
Conserving What for Whom?} (2004)
doi:\href{https://doi.org/10.1111/j.1523-1739.2004.00401.x}{10.1111/j.1523-1739.2004.00401.x}.}

\leavevmode\vadjust pre{\hypertarget{ref-National2020}{}}%
\CSLLeftMargin{15. }
\CSLRightInline{National Conservation Easements Database. {Conservation
Easements Database of the U.S.} (2020).}

\end{CSLReferences}


\end{document}
